\chapter{INTRODUÇÃO}
\label{chap:introducao}

Com a evolução da computação em nuvem, surgiram necessidades cada vez maiores de utilizar ao 
máximo o poder dos computadores sem sobrecarregar os mesmos, e os componentes que o fazem funcionar,
como o uso de energia. Estas necessidades surgem para atender 
requisitos de diminuição de custos, aumento de desempenho, diminuição no consumo de energia,
entre outros objetivos que fazem com que usuários de computação em nuvem e empresas que usam 
este tipo de serviço obtenham vantagem no uso dela.

Para isso, é muito comum que para otimizar o uso dos computadores de um ambiente em nuvem, 
os provedores utilizem o mecanismo de virtualização. Hoje, os \textit{datacenters} são compostos 
por máquinas físicas(PMs) e máquinas virtuais(VMs), sendo que, cada PM normalmente possui 
pelo menos uma ou mais VMs. Essa utilização das VMs permite que seja construído um ambiente flexível
em relação a organização e quantidade de VM por Hosts em cada \textit{datacenter}.

Em um cenário que o ambiente em que temos a possibilidade de utilizar as PMs como host de múltiplas VMs,
é possível que as VMs da nuvem sejam organizadas de diferentes maneiras em relação as PMs. A decisão
de organizar a núvem de uma maneira ou de outra, envolvem os objetivos de quem está gerenciando a núvem. 
 
 Os objetivos podem ser os mais variados. Por exemplo, uma empresa que 
 use o serviço da nuvem pode querer ter um alto desempenho, assim como pode querer ter o menor custo 
 possível. Por esse motivo, existem pesquisas que buscam maneiras de melhorar a alocação e realocação de VMs,
 e uma das formas de resolver este tipo problema, é utilizar a otimização multiobjetivo (MOO). 

A migração de uma VM envolve algumas etapas, como, a descoberta da necessidade de migração, 
a escolha de uma VM a ser migrada e a escolha de um host de destino para essa VM.
A etapa em que este trabalho está interessado é a escolha de um host de destino para a VM. 
Considerando que uma migração seja considerada cara do ponto de vista computacional, 
o momento da migração deve ser bem escolhido para evitar que a migração da VM feita não gere  
prejuízos ou problemas maiores do que a própria sobrecarga do host. 
Assim como o momento da migração é importante, a escolha de um destino também é, 
pois o host selecionado tem que melhorar a maneira em que os hosts estão organizados naquele determinado momento, 
para que não haja novas migrações por conta da migração inicial.

Este trabalho tem papel de servir como apoio para a migração de VMs em ambientes de computação em nuvem. 
Isto é feito através de uma abordagem em que um gerenciador de núvem, que precise migrar uma VM, possa 
utilizar um serviço que selecionará as melhores opções de host para fazer a migração de uma VM. 
O serviço possui uma abordagem que utiliza algoritmos que fazem uma seleção dos melhores hosts baseando-se 
nos objetivos do consumidor do serviço. Contudo, o serviço é uma solução caixa preta, esta característica 
traz uma grande vantagem, o usuário não precisa conhecer nada sobre os algoritmos utilizados, 
precisa apenas utilizar a interface que é definida pelo serviço. A interface do serviço é construida
em cima de \textit{webservices}, utilizando padrões bem conhecidos para facilitar a integração dos gerenciadores
das núvens computacionais.

Este trabalho apresenta uma aplicação prática da construção de um \textit{webservice} utilizando padrões bem conhecidos
na indústria. Além do \textit{webservice}, o trabalho também apresenta uma aplicação de algoritmos de otimização multiobjetivo
para um problema que existe hoje. Também são feitos testes e gerados métricas para a avaliação dos resultados.

\section{ORGANIZAÇÃO DO TRABALHO}
\label{sec:organizacaoTrabalho}

O trabalho está organizado em seis capítulos. O primeiro, destina-se a dar uma introdução do problema e forma de resolvê-lo
Já o segundo capítulo, irá dar o embasamento teórico necessário para o entendimento dos demais capítulos.
O terceiro destina-se a metodologia utilizada para o desenvolvimento do trabalho. No quarto serão apresentados
aspectos da implementação da solução para o problema encontrado. Os resultados obtidos com a implementação
apresentada no capítulo quarto serão feitas no capítulo cinco. E por último as conclusões que foram obtidas
na realização do trabalho.
