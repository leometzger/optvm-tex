\chapter{INTRODUÇÃO}
\label{chap:introducao}

Com a evolução da computação em nuvem, surgiram necessidades cada vez maiores de utilizar ao 
máximo o poder dos computadores sem sobrecarregá-los. Estas necessidades surgem para atender 
requisitos de diminuição de custos, aumento de desempenho, entre outros objetivos que fazem com que 
usuários de computação em nuvem e empresas que usam este tipo de serviço obtenham vantagem no uso dela.

Para isso, é muito comum que para otimizar o uso dos computadores de um ambiente em nuvem, 
os provedores utilizem o mecanismo de virtualização. Hoje, os \textit{datacenters} são compostos 
por máquinas físicas(FMs) e máquinas virtuais(VMs), sendo que, cada FM normalmente possui 
pelo menos uma ou mais VMs. Essa utilização das VMs permite que seja construído um ambiente flexível.

Em um cenário que o ambiente em que temos a possibilidade de utilizar as FMs como host de múltiplas VMs,
 é possível que as VMs da nuvem sejam organizadas de diferentes maneiras em relação as FMs para que atinjam 
 os objetivos dos interessados. Os objetivos podem ser os mais variados. Por exemplo, uma empresa que 
 use o serviço da nuvem pode querer ter um alto desempenho, assim como pode querer ter o menor custo 
 possível. Por esse motivo, existem pesquisas que buscam maneiras de otimizar esses objetivos e buscar 
 uma forma de resolver este tipo problema, o qual é chamado otimização baseado em múltiplos objetivos. 

A migração de uma VM envolve algumas etapas, como, a descoberta da necessidade de migração, 
a escolha de uma VM a ser migrada e a escolha de um host de destino para essa VM.
 A etapa em que este trabalho está interessado é a escolha de um host de destino para a VM. 
 Considerando que uma migração seja considerada cara do ponto de vista computacional. 
 O momento da migração deve ser bem escolhido para evitar que a própria migração não incorra em 
 prejuízos. Assim como o momento da migração é importante, a escolha de um destino também é, 
 pois o host selecionado tem que atender os objetivos e restrições que a VM necessita, para que 
 não haja a sobrecarga do host de destino e implique em uma nova migração.

O trabalho tem papel de servir como apoio para a migração de VMs em ambientes de computação em nuvem. 
O trabalho faz uso de uma abordagem em que um usuário, que precise migrar uma VM, possa 
utilizar um serviço que selecionará as melhores opções de host para fazer a migração de uma VM. 
O serviço possui uma abordagem que utiliza algoritmos que fazem a seleção do host baseado em múltiplos objetivos.
 Contudo, o serviço é uma caixa preta, esta característica traz uma grande vantagem, o usuário não precisa conhecer nada 
 sobre os algoritmos utilizados, precisa apenas utilizar a interface que é definida pelo serviço. 

\section{ORGANIZAÇÃO DO TRABALHO}
\label{sec:organizacaoTrabalho}

@TODO