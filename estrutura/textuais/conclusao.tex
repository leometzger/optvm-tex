\chapter{CONCLUSÃO}
\label{chap:conclusao}
A otimização da seleção de hosts na migração de máquinas virtuais
foi o objetivo da solução deste trabalho. Com o crescimento da 
utilização de computação em nuvem, esse é um tipo de problema em que
a indústria começa a se deparar. A otimização na utilização de recursos 
computacionais tem uma importância por diminuir o custo para empresas,
melhorar o consumo de energia, entre outros fatores que fazem com que 
o trabalho seja útil.

Neste trabalho a solução foi desenvolvida através de um serviço utilizando REST 
como estilo arquitetural e algoritmos genéticos para a otimização. 
O trabalho fornece uma solução genérica, onde, outros trabalhos que 
envolvam migração de máquinas virtuais e que envolvam uma ou mais núvens, 
possam fazer uso da API construída. 

@TODO Apresentar resultados
@TODO Falar sobre resultados

\section{TRABALHOS FUTUROS}
\label{sec:trabalhosFuturos}

Este trabalho permitiu ampliar o campo de visão de soluções para
o problema de otimização de migração de VMs, especialmente na seleção
de hosts. É possível fazer melhores versões ou evoluir a partir deste trabalho.

\subsection{Pesquisas em restrições para o ambiente de núvens federadas}

Neste trabalho, foram apresentadas restrições em nível de \textit{cloud},
\textit{datacenter} e \textit{host}. Futuramente, podem ser exploradas
alternativas para esses níveis. Além disso, também é possível fazer
uma busca de outras restrições, que façam parte de algum destes
três grupos.

\subsection{Comparação de algoritmos MOO para migração de máquinas virtuais}

Algoritmos Genéticos são utilizados neste trabalho para fazer a parte
de MOO. Porém, é possível utilizar vários outros tipos de algoritmos como
\textit{Particle Swarm} (PS), \textit{Variable Neghborhood Search} (VNS) e
\textit{Ant Colony Optimization} (ACO). Diferentes algoritmos podem ser utilizados 
para efeito de comparação para este problema. 

\subsection{Generalização do problema}

O problema abordado neste trabalho é bastante específico e pode ser 
generalizado. Por exemplo, o trabalho não leva em consideração o "efeito colateral"
em razão da migração da VM.

O serviço implementado no trabalho busca responder a pergunta: 
\textbf{"Qual é o melhor subconjunto de hosts para esta VM migrar, respeitando as
restrições e objetivos da política?"}. A partir desse questionamento, uma possível
pergunta para o problema generalizado, levando em considerações os efeitos colaterais,
seria: \textbf{"Qual é a melhor disposição que a federação pode ter considerando
determinado cenário?"}.
 
\section{CONSIDERAÇÕES FINAIS}
@TODO