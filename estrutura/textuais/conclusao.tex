\chapter{CONCLUSÃO}
\label{chap:conclusao}
A otimização da seleção de hosts na migração de máquinas virtuais
foi o objetivo da solução deste trabalho. Com o crescimento da 
utilização de computação em nuvem, esse é um tipo de problema interessa à indústria. 
A otimização na utilização de recursos 
computacionais tem uma certa importância, por diminuir o custo para empresas,
melhorar o consumo de energia, entre outros fatores que fazem com que este
campo de pesquisa mereça atenção.

Neste trabalho a solução foi desenvolvida através de um serviço utilizando REST 
como estilo arquitetural que utiliza algoritmos genéticos para fazer uma otimização, 
além de uma solução própria para a seleção de hosts removendo os que não atendem
certas restrições. 
O trabalho fornece uma solução genérica, onde, outros trabalhos que 
envolvam migração de máquinas virtuais e que envolvam uma ou mais núvens, 
possam fazer uso da API construída, isso tanto em cenários pequenos como em 
cenários maiores.

Os resultados obtidos em relação ao desempenho da API foram satisfatórios.
Os mesmos mostram que o \textit{webservice} escala até certo ponto, diminuindo
seu desempenho conforme aumenta a complexidade do cenário. Inclusive, um dos
diferenciais do OptVM é esta flexibilidade na utilização, apesar da degradação 
do desempenho, continua atendendo.

O trabalho também mostra que, a seleção de hosts para a migração de uma VM,
por se tratar de ser uma tarefa difícil, 
pode valer a pena atribuir o trabalho a um \textit{webservice}, ao invés de lidar 
juntamente com a complexidade de gerenciar o ambiente (hosts/DCs/nuvens). Isso permite
que as responsabilidades sejam separadas mais granularmente.

\section{TRABALHOS FUTUROS}
\label{sec:trabalhosFuturos}

Este trabalho permitiu ampliar o campo de visão de soluções para
o problema de otimização de migração de VMs, especialmente na seleção
de hosts. É possível fazer melhores versões ou evoluir este trabalho.

\subsection{Pesquisas em restrições para o ambiente de núvens federadas}

Neste trabalho, foram apresentadas restrições em nível de \textit{cloud},
\textit{datacenter} e \textit{host}. Futuramente, podem ser exploradas
alternativas para esses níveis. Além disso, também é possível fazer
uma busca de outras restrições, que façam parte de algum destes
três grupos.

\subsection{Comparação de algoritmos MOO para migração de máquinas virtuais}

Algoritmos Genéticos são utilizados neste trabalho para fazer a parte
de MOO. Porém, é possível utilizar vários outros tipos de algoritmos como
\textit{Particle Swarm} (PS), \textit{Variable Neghborhood Search} (VNS) e
\textit{Ant Colony Optimization} (ACO). Diferentes algoritmos podem ser utilizados 
para efeito de comparação para este problema. 

\subsection{Generalização do problema}

O problema abordado neste trabalho é bastante específico e pode ser 
generalizado. Por exemplo, o trabalho não leva em consideração o "efeito colateral"
em razão da migração da VM.

O serviço implementado no trabalho busca responder a pergunta: 
\textbf{"Qual é o melhor subconjunto de hosts para esta VM migrar, respeitando as
restrições e objetivos da política?"}. A partir desse questionamento, uma possível
pergunta para o problema generalizado, levando em considerações os efeitos colaterais,
seria: \textbf{"Qual é a melhor disposição que a federação pode ter considerando
determinado cenário?"}.