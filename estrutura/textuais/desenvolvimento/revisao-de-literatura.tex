\chapter{REVISÃO DE LITERATURA}
\label{chap:fundamentacaoTeorica}

Neste capítulo, são apresentados alguns conceitos e termos utilizados no decorrer do trabalho. 
Estes, também darão base para compreender a escolha das técnicas utilizadas. 
Os conceitos são relacionados a construção do sistema. 
Os conceitos estão relacionados a otimização multiobjetivo e comunicação cliente/servidor, 
que são os dois pilares do sistema que será desenvolvido.

\section{Service Oriented Architecture}
O desenvolvimento de software para um ambiente corporativo é uma tarefa complexa. 
Conforme Brown \cite{brown}, no decorrer dos anos, a comunidade de desenvolvimento de software se 
dedicou em desenvolver novas abordagens, processos e ferramentas para a construção de softwares de 
grande escala. 

Brown considera que uma maneira de descrever um sistema de software é como sendo um composto de 
uma coleção de serviços. Cada serviço, provem um conjunto de funcionalidades bem definidas. 
As funcionalidades do serviço sendo bem definidas e sólidas, torna possível a construção de serviços 
compostos, ou seja, uma funcionalidade que faça a utilização de outras funcionalidades ou serviços. 
Esta modularização e coordenação de serviços e funcionalidades caracteriza um 
Service Oriented Architecture (SOA).

Segundo \cite{valipour}, SOA pode ser definido como um design de software utilizado para 
conectar negócios e recursos computacionais sob demanda, e isso possibilita os usuários 
do serviço (podendo ser outros serviços ou usuários finais) alcançarem seus objetivos. 

Existem diversas maneiras de implementar uma aplicação baseada em SOA, o importante é que 
sua interface seja bem definida com as operações que podem ser realizadas. 
Uma das grandes vantagens do SOA, é a facilidade que ele provém na integração de sistemas. 
Segundo \cite{valipour}, com as operações bem definidas e disponíveis, o consumidor do SOA, 
pode se preocupar somente com o que determinado serviço faz e não como é implementado.

As principais características de um software feito utilizando SOA são que ele é auto contido e 
modular, interoperável, fracamente acoplado, passível de composição e possui transparência de localização. 
Como SOA não limita a estratégia utilizada para o desenvolvimento do mesmo, pode-se utilizar qualquer 
técnica para implementá-lo. No ambiente corporativo, os serviços comumente são implementados 
utilizando web services SOAP, REST ou chamadas RPCs.

\subsection{Simple Object Access Protocol}
SOAP é um protocolo de comunicação baseado em XML(eXtension Markup Language) que foi criado no final 
dos anos 90. Seu objetivo é fazer a comunicação entre o cliente e o servidor através de informações 
passadas através de um documento XML. O protocolo utiliza um \textit{schema} XML, que é uma 
maneira de descrever e validar o formato os dados das requisições e respostas. 
Esse \textit{schema} é utilizado pelo cliente e pelo servidor para saber como interpretar a resposta, 
no caso de recebimento de mensagem, e formatar a requisição, no caso de envio.

O objetivo do SOAP, é expor regras de negócio de aplicação através de serviços. 
Por esse motivo, o SOAP é uma opção comumente utilizada na construção de aplicações SOA. 
Outra característica do SOAP, é que ele não precisa ser implementado sobre um protocolo de 
transporte específico, é possível implementar utilizando outros protocolos, porém, 
na maioria das vezes é utilizado HTTP.

O SOAP algumas vezes é comparado com o REST, pois os dois podem ser utilizados para um objetivo 
semelhante, porém, os dois tem um foco diferente, onde o SOAP tem como expor regras de negócio 
como serviço e o REST visa representar um determinado estado e manipulá-lo através de operações 
bem definidas.

\subsection{Representational State Transfer (REST)}
REST foi formalizado por Fielding \cite{fielding} em sua tese de doutorado, onde ele tem por 
objetivo apresentar uma arquitetura para criação de sistemas network-based . 
Na tese, REST é definido como um estilo arquitetural. Ele define uma série de restrições que devem ser 
respeitadas na criação de um software que é implementado utilizando este estilo arquitetural. 
As características de um software que utiliza o estilo REST serão apresentadas a seguir, assim 
como algumas de suas vantagens de desvantagens.

Uma das principais características é que o REST é implementado utilizando o modelo de comunicação 
cliente-servidor. Isso ajuda com a separação de responsabilidades, e permite que uma portabilidade 
de clientes do serviço implementado. Além disso, essa separação também permite que outros serviços 
façam uso do serviço REST. Do ponto de vista de arquitetura de software, isto é muito importante, 
pois permite que componentes fiquem bem modularizados.

Outra característica que o estilo arquitetural tem, é que o serviços devem ser stateless, ou seja, as  
requisições devem ser auto-contidas, não podem assumir algum estado ou contexto que o servidor tenha 
previamente armazenado. 

Fielding \cite{fielding} destaca que uma das características centrais do REST, e o que difere ele 
de outros estilos arquiteturais, é a utilização de uma interface uniforme entre os componentes. 
Esta interface uniforme é um ponto muito positivo, pois permite que a comunicação entre os 
componentes da arquitetura seja feita de maneira genérica, o que permite escalar a comunicação 
entre aplicações.

\section{Otimização multi-objetivo}  
Conforme Nos dias de hoje, problemas de otimização buscam um bom resultado.
Segundo Veloso \cite{veloso} para os problemas de otimização, existem de maneira geral, 
dois tipos de problemas, os mono-objetivos e os multi-objetivos. 
Os mono-objetivos buscam otimizar uma solução baseando-se em um único objetivo, por consequência, 
problemas mono-objetivo resultam em um único resultado, que pode ser considerada a solução ótima para o 
problema. Já os multi-objetivo, buscam atender vários fatores, e isso torna a solução ótima mais 
difícil de ser encontrada.

Conforme Ticona \cite{ticona} um problema de otimização multi-objetivo, é representado por um 
conjunto de funções objetivo que devem ser otimizadas. 
 
\subsection{Algoritmos Evolucionários}
Segundo \cite{ticona}, algoritmos evolucionários (AE) tem sido muito utilizados para problemas de 
otimização. Um dos motivos do uso deles, é por causa da possibilidade de resolver problemas que 
envolvam múltiplos objetivos. A abordagem utilizada neste tipo de algoritmo é baseada na 
evolução humana. O processo é baseado seleção natural de Darwin, da mesma maneira que acontece com a 
seleção das espécies. O algoritmo reproduz artificialmente o processo de seleção natural para 
encontrar os mais aptos a resolver determinado problema. O objetivo desses algoritmos é 
encontrar aproximações da solução perfeita para problemas difíceis. 

Dentro da categoria dos AEs para otimização baseada em multiplos objetivos, existem diferentes modelos. 

O modelo de utilizado neste trabalho é a de algoritmos genéticos (AG). Esta é uma classe de 
algoritmos muito utilizada em otimizações multi-objetivo.

\subsection{Algoritmos Genéticos}
Existem duas abordagens principais para AGs multi-objetivo. Uma delas utiliza pesos para objetivos 
únicos. E a outra forma, seleciona um subconjunto de um conjunto de possíveis soluções que não é 
dominada por nenhuma das outras soluções, este subconjunto é chamado de lista de Pareto, 
também conhecida por soluções Pareto-ótimas.

Nas otimizações multi-objetivo que utilizam soluções pareto-ótimas, usa-se o conceito de 
dominância de Pareto para  para alcançar soluções que sejam mais  adequadas para determinado problema. 
Segundo Ticona\cite{ticona}, a dominância de um item \textbf{x} sobre um item \textbf{y} se dá quando 
as seguintes condições são atendidas:

\begin{enumerate}
\item A solução \textbf{x} é igualmente ou mais adequada que a solução \textbf{y} em todas as funções objetivo
\item A solução \textbf{x} é melhor do que a solução \textbf{y} em algum objetivo
\end{enumerate}

\section{Trabalhos relacionados}
Neste capítulo será apresentado o problema em que o OptVM se propoem resolver. 
Isso será feito através da apresentação de alguns fatos relacionados a nuvens computacionais 
e como elas costumam ser utilizadas nos dias de hoje.

\subsection{Migração de máquinas virtuais}
As nuvens computacionais são utilizadas pela maioria das empresas de software da atualidade. 
Por esse motivo, as maiores empresas do setor investem muito neste segmento, 
oferecendo vários tipos de serviços diferenciados para seus consumidores. 
Estas empresas concorrem em alguns aspectos, como: velocidade, preço, disponibilidade e etc. 
Para aumentar sua competitividade nesses aspectos, muitas vezes é utilizada a virtualização.

Com a virtualização é possível alocar partes de um recurso físico para diferentes consumidores, 
fazendo com que um recurso físico se torne melhor utilizado. Isso deixa a alocação de recursos
muito mais flexível, e torna possível obter uma elasticidade nos serviços oferecidos. 

Uma dos objetivos de utilizar a virtualização é obter uma elasticidade dos recursos oferecidos.
Ou seja, é possível aumentar sua capacidade de processamento, armazenamento mesmo depois que o
já foi alocada uma VM para o usuário. 
Isso permite que um consumidor do  serviço possa escolher o quanto precisa para executar as tarefas que deseja, 
assim como o provedor também consegue otimizar o uso de seus recursos

Essa realocação de uma VM pode ser feita a nível de nuvem, datacenter(DC) ou host. 
Quando a realocação é feita em nível de DC ou nuvem, é muito provável que seja 
necessário migrar uma VM do local em que ela se encontra. 
Caso seja necessário uma migração de uma máquina, existem alguns pontos que devem ser avaliados.

Três momentos podem ser considerados os pontos principais a serem avaliados para uma migração, 
são eles:

\begin{enumerate}
\item A descoberta de uma necessidade de migração
\item Qual máquina virtual deve ser mgirada
\item Para onde deve ocorrer a migração
\end{enumerate}

Estas otimizações e migrações das VMs são necessárias em ambientes que envolvem uma infraestruturas
grande, onde existem multiplos hosts, datacenters e nuvens. Por esse motivo, não é recomendado que 
um sistema ou serviço gerencie a infraestrutura inteira sozinho, pois sua escalabilidade 
poderia se tornar um gargalo. Isso faz com que sejam construidos diferentes serviços e aplicações 
que se integram e gerenciam a infraesturua.

\subsection{Otimização na escolha do host}
Como citado anteriormente uma das partes essenciais na migração de uma VM é a escolha de um destino para ela.
Para isso, é importante escolher um destino que aloque muito bem a VM e não seja necessário fazer uma outra
migração logo em seguida. 
