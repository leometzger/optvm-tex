\chapter{OPTVM}

O OptVM é um sistema que tem o propósito de dar suporte para a migração de VMs através de serviços utilizando o modelo cliente/servidor. 
O sistema possui dois principais serviços para atingir seu objetivo: 
um faz a filtragem de hosts aplicando restrições definidas pelo cliente do serviço e a 
outra destina-se a quem precisa definir os melhores hosts para migrar uma VM baseando-se em 
objetivos também definidos pelo cliente.

Por lidar com conhecimentos específicos, o OptVM busca ser uma solução caixa preta, onde, o usuário não necessita 
saber nada sobre o funcionamento interno, algoritmos utilizados, etc. Basta utilizar suas APIs para fazer uso de suas funcionalidades.

Neste capítulo, serão apresentadas uma visão geral da a implementação do OptVM. 
No primeiro momento será falado sobre o modelo de comunicação que foi escolhido e o porquê. 
Após isso, técnicas e ferramentas utilizadas para a construção dos serviços de aplicação das constraints e também da otimização.

\section{Comunicação}
Em termos gerais, uma API é uma interface de software que pode se

Como o OptVM é um serviço que deve ser disponibilizado para uma arquitetura de 
cliente/servidor de maneira distribuída, haviam três possíveis maneiras de implementá-lo, 
que eram REST, SOAP e via chamadas RPC. 
Para o desenvolvimento do OptVM o foi escolhido implementação utilizando o modelo REST. 
A escolha desta opção se deu pelos seguintes motivos:

\begin{enumerate}
\item É um padrão arquitetural bastante maduro;
\item É agnóstico em relação a liguangens de programação;
\item É bastante flexivel em relação ao modelo de comunicação.
\end{enumerate}

O padrão REST, definido por Fielding, sugere que se deve criar uma interface para interação com o sistema. Essa interface é representada
através de recursos.

O OptVM trabalha em cima de um único recurso, chamado \textit{otpimizations}.

\section{Representação do serviço}

O padrão arquitetural REST é agnostico em relação ao formato utilizado para fazer a comunicação dos dados. O \textit{econding} 
dos dados pode ser feito da maneira que for mais conveniente para o usuário. No caso do OptVM

\begin{lstlisting}[language=json,firstnumber=1]
{
  "id": 1,
  "objectives": [
    "MIN_SOMETHING",
    "MAX_SOMETHING"
  ],
  "hosts": [
    {
      "id": 1,
      "memory": 2000,
      "bandwidth": 1000,
      "vms": [
        {
          "id": 10,
          "space": 20,
          "memory": 150
        }
      ]
    }
  ]
}
\end{lstlisting}

\section{Representação para o algoritmo}

Nos algoritmos genéticos, existem vários tipos de representações que podem ser utilizadas para representar nossa solução. Quando utiliza-se
o algoritmo, uma boa representação ajuda a melhorar os resultados do algoritmo. 

A representação escolhida para esse serviço foi a representação por inteiros, onde, cada inteiro representa um host e 
a posição em um vetor de inteiros representa uma VM.

A primeira posição "0" significa que a VM 0 está presente no HOST 10, a VM 1 está presente no HOST 20, e assim sucessivamente.

\section{Utilização do algoritmo}

TODO