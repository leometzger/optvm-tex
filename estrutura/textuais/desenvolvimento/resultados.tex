\chapter{RESULTADOS}
\label{chap:results}

Este capítulo é dedicado à avaliação dos resultados obtidos com testes
feitos utilizando a solução desenvolvida no trabalho. Os testes
utilizaram um ambiente de VM para que os testes possam ser reproduzidos. 
Foram utilizadas diferentes métricas para avaliar
o comportamento da API. Além das diferentes métricas, os dados inseridos 
também possuem diferentes formatos e valores. Isso foi feito para que 
seja avaliado o comportamento da API em diferentes cenários, o objetivo é perceper
se o comportamento é alterado considerávelmente ou mantem seu comportamento
mesmo em casos extremos.

Esta avaliação é importante, pois,
dependendo aonde a API estiver instalada, cada requisição gera um custo e este custo 
deve ser conhecido. Por exemplo, se a API estiver hospedada em nuvem, cada requisição tem 
um custo, seja ele por tráfego de rede, armazenamento dos dados, processamento, etc. 
As métricas utilizadas buscam auxiliar na decisão de utilização da API.

Para a avaliação da API, as métricas utilizadas foram a quantidade de requisições por segundo (RPS)
que a API atende, tamanho do payload (PAYsize) e tempo de execução na fase de otimização (OTE).

Os testes realizados foram aplicados somente no recurso de otimizações,
que é a seleção de \textit{hosts}. Apesar da API ter outros recursos, como o de políticas,
este recurso é apenas um suporte para flexibilizar o uso das migrações. O problema 
que o trabalho se propõe a resolver envolve somente a etapa de criação de uma
otimização. Sendo assim, este é o cenário que é avaliado. Os testes utilizaram um 
ambiente controlado (uma VM com recursos limitados/conhecidos) e utilizaram ferramentas para 
adquirir os dados relativos ao desempenho da API.

Após os testes, foi analisado os dados adquiridos e obtido conclusões a partir deles.
Nas próximas seções, é detalhado como foi montado o ambiente, 
como foi implementada a maneira de obtenção de dados para os testes,
ferramentas utilizadas para obter as métricas e por último os resultados das comparações 
para a obtenção das conclusões.

\section{AMBIENTE}

O ambiente utilizado foi uma máquina virtual utilizando \textbf{Ubuntu Server 18.04}
com memória de \textbf{4.0GB} utilizando um processador \textbf{2,3 GHz Intel Core i5}. 
As requisições são feitas através de uma rede \textit{bridge} da máquina física para a virtual.

A escolha de um ambiente com VM é permitir o isolamento de recursos. Além disso,
o ambiente é mantido de forma mais controlada, sendo assim, há uma melhor 
consistencia dos testes feitos. Além do maior controle sobre o ambiente, é importante salientar 
que o ambiente simulado está disponível nas nuvens comerciais na indústria, como AWS, AZURE e 
etc. Sendo assim, caso uma máquina com as características da VM testada for utilizada, 
o desempenho da API tende a ser o mesmo.

\section{MÉTRICAS UTILIZADAS}
É muito comum uma API ser medida através de uma medida quantitativa das requisições 
por segundo (RPS) que ela consegue atender. O número de requisições 
varia de aplicação para aplicação, e isso depende da tecnologia utilizada, 
arquitetura do sistema, se há consulta em banco de dados ou não, algoritmos utilizados, etc.
Esta métrica define se a utilização de um serviço é viável em relação a quantidade de consumidores ou
quantidade de utilização, ou seja, se atende as necessidades de uso do consumidor do serviço. 
Através desta métrica é possível identificar gargalos na utilização. Para esta métrica, foi utilizada 
a ferramenta WRK. O WRK é uma ferramenta \textit{open-souce} construída em C utilizada 
para fazer \textit{benchmarking} de aplicações HTTP. Com ela é possível fazer requisições 
concorrentes utilizando múltiplos ou um único core. Além disso, o WRK permite obter uma duração
do teste, threads e conexões HTTP.

Além da métrica de requisições por segundo, outro aspecto importante na utilização de API
é o tamanho do \textit{payload}. O payload, muitas vezes é atribuido um custo por 
tráfego da máquina, e por processamento e armazenamento, sendo assim, quanto maior
o payload, mais tráfego, processamento e armazenamento ele gerará, consequentemente,
aumentará o custo. Esta métrica pode identificar possíveis erros em relação 
ao modelamento da API, ou seja, a API modelada de uma maneira mais otimizada, utilizando os dados
de maneira diferente, pode obter um \textit{payload} menor.

O OptVM salva o tempo de execução de cada etapa da criação de uma otimização, que são, 
a aplicação das constraints e MOO. Por ser a fase mais significativa, em relação ao processamento,
na avaliação dos resultados quantitativos, foi considerada a média de tempo
de execução da MOO. A esta métrica foi dado o nome de \textit{optimization execution time}(OET).
Esta métrica permite identificarmos o impacto da fase de MOO na requisição.

\begin{table}[!htb]
    \centering
    \caption[Métricas]{Tabela das métricas
    \label{tab:response-host}}
    \begin{tabular}{|m{3cm}|m{7cm}|m{4cm}|}
        \toprule
            Métrica & Objetivo & Ferramenta \\ 
        \midrule
            RPS & Medir desempenho da aplicação & WRK \\
            PAYsize & Medir quantidade de dados trafegados & Tamanho do JSON \\
            Erros & Verificar se ocorrem erros & WRK \\
            OET & Verificar se é o causador da medida & Dados do OptVM \\
        \bottomrule
    \end{tabular}
\end{table}



\subsection{WRK}

Como o resultado de um teste utilizando o WRK consegue nos trazer algumas informações 
sobre a API, como RPS, transferência por segundo, entre outras. Das informações que
o WRK nos traz, as escolhidas e o seu respectivo motivo foram:

\begin{enumerate}
  \item RPS: Medir a velocidade do serviço;
  \item Respostas com status de insucesso: Medir a confiabilidade. 
\end{enumerate}

Para o teste de RPS da API, foram utilizados os seguintes valores para os parâmetros do WRK:

\begin{enumerate}
  \item Conexões: 25
  \item Threads: 1
  \item Tempo de teste: 3 minutos
\end{enumerate}

Os parâmetros escolhidos para a realização dos testes, significam que uma thread está disparando requisições através 
de 25 conexões para o serviço ou seja, o serviço recebe as requisições através de 25 conexões
concorrentes. Já o tempo escolhido, foi de três minutos, para que não haja um viés no número
de requisições por conta de alguma lentidão inicial, por exemplo, pela abertura de conexão com 
banco de dados.

\section{DADOS UTILIZADOS}
A API do OptVM permite que sejam feitas várias combinações de dados. As combinações
podem variar em relação à restrições utilizadas, objetivos da otimização, quantidade
de hosts, nuvens e datacenters envolvidos, entre outros. Isso é importante pois 
permitem avaliar diferentes cenários, ou seja, diferentes combinações, simulando diferentes
casos de uso da API.

Na avaliação dos resultados, foram utilizados dados que simulam casos simples (utilizando poucos
hosts, restrições e objetivos) e casos mais complexos (utilizando mais hosts, restrições e objetivos). 
O formato dos cenários utilizados é demonstrado na Tabela \ref{tab:testdata-info}

\begin{table}[!htb]
    \centering
    \caption[Formato dos dados de teste]{Tabela de formato dos dados de teste
    \label{tab:testdata-info}}
    \begin{tabular}{rrrrr}
        \toprule
            Ambiente & Restrições & Objetivos & Clouds & Hosts \\ 
        \midrule
            Simples & 0 & 2 & 1 & 100 \\
            Médio & 2 & 3 & 2 & 500 \\
            Complexo & 4 & 3 & 5 & 2500 \\
        \bottomrule
    \end{tabular}
\end{table}

Os dados utilizados foram gerados através de um código java. O código
construído gerou em todos os casos 5 nuvens, sendo que, a quantidade de
datacenters/hosts varia para ficar compatível com cada cenário. Para cada
host, atribuiu-se 5 VMs.

Para os testes conter os dados mais próximos da realidade quanto possível,
foram pré-definidos alguns valores para atributos que são fixos, por exemplo,
sistema operacional, memória RAM total, armazenamento total. Nestes casos, um dos valores
pré-definidos foi selecionado aleatoriamente como valor para o atributo da entidade. 
Nos casos onde o atributo é completamente variável, foram atribuídos valores aleatórios para os 
mesmos mantendo um intervalo condizente com a realidade, por exemplo, preço por armazenamento
entre 0.1 e 1 (centavos).

Como o OptVM permite escolher o tamanho do subconjunto de \textit{hosts} que quer obter como resposta,
além dos dados da nuvem, DCs e hosts, foi utilizado como padrão a busca de um subconjunto de
3 hosts como sendo uma possível solução. Essa quantidade de hosts foi escolhida pois em um ambiente 
real, caso seja escolhido um único host, o mesmo pode não estar disponível no momento da migração.

\section{COMPARAÇÕES}

Nesta seção são apresentados os comparativos entre os cenários testados, utilizando os dados, 
métricas e ambiente descritos nas seções anteriores. 

Para as métricas, os seguintes resultados foram obtidos:

\begin{table}[!htb]
    \centering
    \caption[Resultado \textit{payload}]{Tabela de tamanho do Payload 
    \label{tab:result-payload}}
    \begin{tabular}{rrrrr}
        \toprule
            Cenário & Tamanho Payload(KB) & RPS & Erros & OET(ms) \\ 
        \midrule
            Super Simples & 85 & 11,25 & 0 & 147,37 \\
            Simples & 419 & 8,11 & 0 & 149,04 \\
            Médio & 836 & 5,95 & 23 & 186,98 \\
            Difícil & 2000 & 4,17 & 12 & 205,60 \\
        \bottomrule
    \end{tabular}
\end{table}

Na comparação dos cenários, pôde ser observado os resultados para os diferentes cenários:
o \textbf{Super Simples}, \textbf{Simples}, \textbf{Médio} e \textbf{Difícil}. 
Foram encontradas significativas diferenças nos resultados. A partir destes resultados 
algumas conclusões podem ser tiradas:

\begin{enumerate}
  \item Quando a requisição fica maior, eventuais erros começam a ocorrer, como nos cenários 
  \textbf{Médio e difícil};
  \item A quantidade de RPS diminui significativamente por causa do tamanho do \textit{payload};
  \item A fase de MOO não é afetada drasticamente nos diferentes cenários, apesar do 
  significativo aumento de \textit{hosts}. Isso significa que não é o gargalo do serviço.
\end{enumerate}

Quando comparado ao número de requisições de outros \textit{webservices}, a quantidade de RPS
é relativamente baixa. Porém, para o propósito deste \textit{webservice} a quantidade pode ser
suficiente, pois, como a migração é um evento custoso computacionalmente, e o ideal é evitá-la, 
consequentemente, poucas requisições são feitas e busca-se diminuir a quantidade ao decorrer do tempo.

Um dos diferenciais do OptVM é a flexibilidade que ele dá ao usuário da API para configurar
a otimização da maneira que melhor se adequar ao seu problema, através das políticas e de 
parâmetros. Tanto um usuário do \textit{webservice} que gerencie um ambiente complexo,
quanto um que gerencie um ambiente simples, podem obter vantagens o utilizando.
