\chapter{RESULTADOS}
\label{chap:results}

Este capítulo é dedicado aos resultados obtidos com testes
feitos utilizando a solução desenvolvida no trabalho. Os testes
utilizaram um ambiente de VM para que os testes possam ser reproduzidos 
mais fácilmente. Foram utilizadas diferentes métricas para avaliar
o comportamento da API. Além das diferentes métricas, os dados inseridos 
também possuem diferentes formatos e valores. Isso foi feito para que 
seja percebido o comportamento da API em diferentes cenários, assim como
avaliar se o comportamento é alterado drásticamente ou mantem seu comportamento
mesmo em casos extremos.

É importante definir bem como a API se comporta em cenários diferentes, pois,
dependendo aonde ela estiver instalada, cada requisição gera um custo. Por exemplo, se a API 
estiver hospedada em nuvem, cada requisição tem um custo, seja ele por tráfego de
rede, armazenamento dos dados, processamento, etc. 

\section{AMBIENTE}

O ambiente utilizado foi uma máquina virtual utilizando \textbf{Ubuntu Server 18.04}
com memória de \textbf{4.0GB} utilizando um processador \textbf{Intel i5}. As requisições
são feitas através de uma rede \textit{bridge} da máquina física para a virtual.
A escolha de um ambiente com VM é permitir o isolamento e reserva de recursos.

\section{DADOS UTILIZADOS}
A API do OptVM permite que seja feitas várias combinações de dados. As combinações
podem variar em relação à restrições utilizadas, objetivos da otimização, quantidade
de hosts, nuvens e datacenters envolvidos, entre outros. 

Na avaliação dos resultados, foram utilizados dados que simulam casos simples e casos mais 
complexos. O formato dos cenários utilizados é demonstrado na Tabela \ref{tab:testdata-info}

\begin{table}[!htb]
    \centering
    \caption[Formato dos dados de teste]{Tabela de formato dos dados de teste
    \label{tab:testdata-info}}
    \begin{tabular}{rrrrr}
        \toprule
            Ambiente & Restrições & Objetivos & Hosts \\ 
        \midrule
            Super Simples & 0 & 2 & 100 & \\
            Simples & 1 & 2 & 500 \\
            Médio & 2 & 3 & 1000 \\
            Difícil & 3 & 3 & 2500 \\
        \bottomrule
    \end{tabular}
\end{table}

\section{MÉTRICAS UTILIZADAS}
É muito comum uma API ser medida através de uma medida quantitativa de requisições 
por segundo (RPS) que ela consegue atender. Porém, o número de requisições 
varia, e isso depende da tecnologia utilizada, arquitetura do sistema, se há 
consulta em banco de dados ou não, algoritmos utilizado, etc.

Além da métrica de requisições por segundo, outro aspecto importante na utilização de API
é o tamanho do \textit{payload}. O payload, muitas vezes é atribuido um custo por 
tráfego da máquina, e por processamento e armazenamento, sendo assim, quanto maior
o payload, mais tráfego, processamento e armazenamento ele gerará, consequentemente,
aumentará o custo. 

\subsection{WRK}

O WRK é uma ferramenta construída em C utilizada para fazer \textit{benchmarking} de 
aplicações HTTP. Com ela é possível fazer requisições concorrentes utilizando 
múltiplos ou um único core.

\section{COMPARAÇÕES}
@TODO