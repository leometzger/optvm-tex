\begin{resumo}[RESUMO]
\begin{SingleSpacing}

\imprimirautorcitacao. \imprimirtitulo. \imprimirdata. \pageref {LastPage} f. \imprimirprojeto\ – \imprimirprograma, \imprimirinstituicao. \imprimirlocal, \imprimirdata.\\

A migração de máquinas virtuais (VMs) é uma necessidade em ambientes 
de federação de nuvens computacionais. Isso acontece porque as empresas
buscam utilizar o melhor possível os recursos disponíveis em sua infraestrutura.
Porém, gerenciar as migrações de uma federação não é uma tarefa simples. O gerenciamento
envolve pelo menos três etapas, como: identificar que uma VM deve ser migrada, determinada
qual VM deve ser migrada e achar um destino para a VM. Este trabalho serve como apoio 
para a terceira etapa, a seleção de um destino para a VM. A escolha de hosts de destino pode conter
diferentes variáveis, restrições e objetivos. Tendo isso em mente, este trabalho apresenta uma solução
baseada em \textit{webservice} chamada de OptVM que serve de apoio para a migração, com o foco na seleção 
de \textit{hosts} de destino para uma VM. Como a escolha de um destino envolve 
vários objetivos, o OptVM utiliza uma classe de algoritmos de otimização baseado em 
múltiplos objetivos para fazer a escolha do host. O OptVM também fornece uma série de restrições
que servem como um filtro dos hosts que não atendem a regras de negócio específicas do usuário.
Além disso, utiliza padrões conhecidos para a construção do \textit{webservice}.  
A solução proposta também é bastante flexível, permitindo que o 
usuário do \textit{webservice} configure as restrições de acordo com sua necessidade e 
selecione somente os objetivos de interesse.

\textbf{Palavras-chave}: Migração. Otimização multiobjetivo. REST.

\end{SingleSpacing}
\end{resumo}
